%%% Local Variables:
%%% mode: latex
%%% TeX-master: t
%%% End:
\documentclass{beamer}
\usepackage[utf8]{inputenc}
\usepackage{graphics}
\usepackage{listings}
\usepackage{caption}

\captionsetup{font=scriptsize,labelfont=scriptsize}

\usetheme{default}
\usecolortheme{rose}

\DeclareGraphicsRule{.pdftex}{pdf}{.pdftex}{}

% \lstdefinelanguage{cfengine}
%   {morekeywords={import,classes,control,admit,copy,editfiles,processes,shellcommands},
%    sensitive=false,
%    morecommment[l]{//},
%   }

\newcommand\Fontvi{\fontsize{6}{7.2}\selectfont}

\lstset{
basicstyle=\tiny,
stringstyle=\tiny,
numbers=left,
numberstyle=\tiny,
stepnumber=2,
frame=single,
%language=cfengine,
captionpos=b
}

\title{Continuously deliver your puppet code with jenkins, r10k and git\\}
\author{Toni Schmidbauer}

\begin{document}

\begin{frame}
  \center\includegraphics[height=2.5cm,width=7cm]{../pics/puppetcamp_300dpi}
  \titlepage
\end{frame}

\begin{frame}
  \frametitle{whoami}
  \begin{itemize}
  \item SysAdmin@s-itsolutions.at
  \item toni@stderr.at
  \item stderr@jabber.org
  \item \url{http://stderr.at}
  \item \url{http://github.com/tosmi}
  \end{itemize}
\end{frame}
\begin{frame}

  \frametitle{Agenda}

  \begin{itemize}
  \item A short story about configuration management
  \item What is continuous delivery
  \item Tools used to achieve continuous delivery
  \item DEMO
  \item Things to improve
  \end{itemize}

\end{frame}

\begin{frame}
  \frametitle{A short story about configuration management (CM)}

  \begin{itemize}
  \item<1-> We manage a very diverse environment of UNIX/Linux Systems (Solaris 10/11 SPARC/i386, AIX, RHEL/CentOS 5/6/7)
  \item<2-> We've got around 1000 nodes in total
  \item<3-> Before CM we had \textbf{strict} standards on how to manage these systems
  \item<4-> The problem: \\count(team members) == count(standards)
  \item<5-> So configuration management is the solution to all our problems
  \end{itemize}

\end{frame}

\begin{frame}
  \frametitle{The solution to all our problems}
  \pause
  \begin{itemize}
  \item Broke our systems
  \end{itemize}
\end{frame}

\begin{frame}
  \center{\huge{WHY????}}
\end{frame}

\begin{frame}
  \frametitle{Problems with our old CM system}

  \begin{itemize}
  \item <1-> Deployments sucked
    \begin{itemize}
    \item <2-> Deployment via manual tagging and checkout, so mistakes happened
    \item <2-> Deployment in stages, but we always had to cross our fingers
    \end{itemize}
  \item <3-> Testing sucked
    \pause
    \begin{itemize}
    \item <4-> No Unittests
    \item <4-> No acceptance tests
    \end{itemize}
  \item <5-> No immediate feedback if things where OK or \textbf{not}
  \item <6-> Systems installed without CM are hard to bring under CM control
  \item <7-> Every system was a special case
  \end{itemize}
\end{frame}

\begin{frame}
  \center{\huge{So what's our solution?}}
\end{frame}


\begin{frame}
  \frametitle{Continuous delivery}
  \begin{itemize}
  \item<1-> is a pattern for getting software from development to release
  \item<2-> this pattern is called \textbf{the deployment pipeline}
  \end{itemize}
  \footnote{Continuous Delivery: Jez Humble, David Farley}
\end{frame}

\begin{frame}
  \frametitle{The deployment pipeline}
  \begin{figure}[ht]
    \centering
      \includegraphics[height=1.2cm,width=11.5cm]{../pics/deployment_pipline}
    \label{fig:stack}
  \end{figure}
\end{frame}

\begin{frame}
  \frametitle{Why are we using continuous delivery}

  \begin{itemize}
  \item<1-> When you automate your deployment,
    \begin{itemize}
    \item<2-> less mistakes will happen and the same mistake will not happen twice
    \item<3-> less mistakes means less stress when deploying
    \item<4-> less stress means you are going to deploy more often
    \item<5-> more deployments means a more flexible environment
    \end{itemize}
  \end{itemize}
\end{frame}

\begin{frame}
  \begin{figure}[ht]
    \centering
      \includegraphics[height=7.5cm,width=6cm]{../pics/cd_book}
    \label{fig:stack}
  \end{figure}
\end{frame}

\section{Tools we used to build a deployment pipeline}
\begin{frame}
  \center \huge Tools to build a deployment pipeline
\end{frame}


\begin{frame}
  \frametitle{Jenkins}

  \begin{itemize}
  \item Jenkins is an Open Source continuous integration server
  \item It's purpose is to execute and monitor jobs
  \item Jobs are shell scripts or any other thing that's executable
    and returns 0 on success
  \item Many plugins available to extend Jenkins (e.g. git, build-pipeline, monitor)
  \item You can \textbf{link jobs} together, thats our \textbf{pipeline}
  \end{itemize}

\end{frame}


\begin{frame}
  \frametitle{Jenkins II}
  \begin{figure}[ht]
    \centering
      \includegraphics[height=3.5cm,width=11.5cm]{../pics/jenkins_pipeline}
    \label{fig:stack}
  \end{figure}
\end{frame}


\begin{frame}
  \frametitle{Monitoring with Jenkins}
  \begin{figure}[ht]
    \centering
      \includegraphics[height=6cm,width=11cm]{../pics/jenkins_monitor_live}
    \label{fig:stack}
  \end{figure}
\end{frame}

\begin{frame}
  \frametitle{GIT}

  \begin{itemize}
  \item<1-> \textbf{git push} triggers the deployment pipeline
  \item<2-> one central repository for internal modules
  \item<2-> gitolite for access control
  \item<2-> 3 main branches
    \begin{itemize}
    \item development
    \item testing
    \item production
    \end{itemize}
  \item<2-> feature branches for new site local modules
  \item<2-> hiera data is in the same repository
  \end{itemize}
\end{frame}

\begin{frame}
  \frametitle{GIT repository layout}

  \begin{itemize}
  \item \texttt{modules/}:
    \begin{itemize}
    \item where r10k stores external (forge, github) modules
    \end{itemize}
  \item \texttt{site/}:
    \begin{itemize}
    \item site local modules, we do not want to share
    \end{itemize}
  \item \texttt{hiera/}:
    \begin{itemize}
    \item our hiera yaml files
    \end{itemize}
  \item \texttt{Puppetfile}:
    \begin{itemize}
    \item config file for r10k that specifies which external modules we need
    \end{itemize}
  \item \texttt{Vagrantfile}:
    \begin{itemize}
    \item To boot a development puppet environment on your local workstation
    \end{itemize}
  \end{itemize}
\end{frame}


\begin{frame}
  \frametitle{GIT workflow}
  \begin{figure}
    \centering
      \includegraphics[height=6cm,width=11cm]{../pics/puppet_deployment2}
    \label{fig:stack}
  \end{figure}

\end{frame}

\begin{frame}
  \frametitle{r10k}

  \begin{itemize}
  \item a tool to deploy puppet environments and modules
  \item every git branch gets deployed to a corresponding puppet environment
  \item it also downloads and installs modules from puppetforge or github
  \item in the current version (1.3.2) dependencies have to be managed
    manually
  \end{itemize}
\end{frame}

  \begin{frame}[fragile]
    \frametitle{Example Puppetfile}
\begin{lstlisting}
forge 'forge.puppetlabs.com'

mod 'puppetlabs/ntp', '3.1.2'
mod 'puppetlabs/postgresql', '3.4.2'
mod 'puppetlabs/stdlib', '4.3.2'
mod 'puppetlabs/firewall', '1.1.3'
mod 'puppetlabs/apache', '1.1.1'
mod 'puppetlabs/lvm', '0.3.2'
mod 'nosolutions/tsm', '0.2.2'
mod 'saz/sudo', '3.0.6'
mod 'spiette/selinux', '0.5.4'

mod 'concat',
    :git => 'https://github.com/puppetlabs/puppetlabs-concat',
    :commit => 'feba3096c99502219043b8161bde299ba65e7b8a'
\end{lstlisting}

    You are able to pin to a git tag / branch / commit hash

\end{frame}

\begin{frame}
  \frametitle{a word on testing}

  \begin{itemize}
  \item you must have unit tests for your puppet code: \textbf{rspec-puppet}
  \item for acceptance tests there's \textbf{puppetlabs/beaker}
  \item you need to test everything to get most out of the build
    pipeline
  \item we test
    \begin{itemize}
    \item internal puppet modules
    \item hiera data
    \item puppet configuration
    \end{itemize}
    \item all internal modules are required to have rspec tests
  \end{itemize}
\end{frame}

\begin{frame}[fragile,fragile]
  \frametitle{rspec-puppet example}

  samplemodule/manifests/init.pp

\begin{lstlisting}
class samplemodule ( $message = 'defaultmessage' ) {
  notify { 'samplemessage':
    message => "This is the sample module, my message is: $message",
  }
}
\end{lstlisting}

  samplemodule/spec/classes/samplemodules\_spec.rb

  \begin{lstlisting}
require 'spec_helper'

describe 'samplemodule', :type => :class do
  context 'with default parameters' do
    it { should contain_notify('samplemessage') }
  end
end
  \end{lstlisting}

\end{frame}

\begin{frame}[fragile]
  \frametitle{beaker example}

\begin{lstlisting}
require 'spec_helper_acceptance'

describe 'profiles::ossbase class' do
  it 'should work with no errors' do
    apply_manifest('include profiles::ossbase', :catch_failures => true)
  end
end
\end{lstlisting}

\end{frame}

\begin{frame}
  \center{\huge DEMO}
\end{frame}

\begin{frame}
  \frametitle{Do try this at home}

  \begin{itemize}
  \item You need:
    \begin{itemize}
    \item Vagrant from \url{http://vagrantup.com}
    \item Virtualbox
    \item Git client
    \end{itemize}
  \item You have to run:
    \begin{itemize}
    \item \texttt{git clone https://github.com/tosmi/puppetcamp2014.git}
    \item \texttt{cd puppetcamp2014}
    \item \texttt{vagrant up}
    \item \texttt{vagrant ssh}
    \end{itemize}
  \end{itemize}
\end{frame}

\begin{frame}
  \frametitle{Things we have to improve}

  \begin{itemize}
  \item We need more test Systems (CentOS/RHEL/Solaris/AIX?)
  \item We need more acceptance tests
  \item Once again use stages in production
  \item Our documentation
  \end{itemize}
\end{frame}

\begin{frame}
  \frametitle{Things puppetlabs should improve }
  \begin{itemize}
  \item Puppetlabs should package beaker as a rpm/deb whatever, gems
    suck in production
  \end{itemize}

\end{frame}

\begin{frame}
  \frametitle{Summary}

  \begin{itemize}
  \item<1-> Continuous Delivery is implemented via a deployment pipeline
  \item<2-> Within the pipeline \textbf{everything} is automated
  \item<3-> When everything is automated you need to test everything
  \item<4-> When everything is automated and well tested deploying becomes easy
  \item<5-> Tools we are using to implement a deployment pipeline
    \begin{itemize}
    \item Jenkins: to execute jobs
    \item GIT: version control
    \item Gitolite: access control and authorization for GIT
    \item r10k: install puppet modules / create puppet environments from branches
    \item rspec-puppet: unittests for puppet
    \item puppetlabs/beaker: acceptance tests for puppet
    \end{itemize}
  \end{itemize}
\end{frame}

\begin{frame}
  \center{\huge Thanks for your attention!}
\end{frame}

\end{document}
